%!TEX root = main.tex
\section{Discussion}
% \paragraph{Distributed Training}
\setlength{\parindent}{2em}\textbf{\textit{a) Distributed training:}}
Distributed training is common to speedup the overall training process
and support massive models, such as GPT-serial models~\cite{radford2019language,brown2020language}.
It includes data parallelism, model parallelism~\cite{shoeybi2019megatron}, and pipeline model parallelism~\cite{ghuangGpipeEfficientTraining2019,narayanan2019pipedream,eliad2021fine}.
% But the is much difficult compared to singe-GPU training.
The distributed training is complicated than single-GPU training
as it introduces communication and parameters aggregation across GPUs.
Nonetheless, the memory usage of an iteration inside a single GPU still follows \emph{``first increase then drop"} pattern.
Therefore, the idea in \oursys{} is still applicable in distributed training.
% The concurrent training on distributed training ,
But the selection of jobs to co-locate with distributed training job needs to be more careful,
since the speed slowdown on one GPU lead to the performance degradation of overall distributed training.
% one slowdown on a GPU will 
We leave the extension of \oursys{} to distributed training as a future work.

\textbf{\textit{b) Job priority:}}
% \paragraph{Job priority}
There are usually two kinds of jobs in DL GPU cluster: \emph{high-priority job} and \emph{low-priority job}.
When co-locating jobs with different priorities,
it's an important and challenging problem to provide performance guarantee on jobs with high-priority.
AntMan~\cite{xiao2020antman} made an attempt to achieve it through limiting the operator launch rate of low-priority jobs.
The current scheduling in \oursys{} does not take the job priority into account.
Nonetheless, it's easy to integrate the batch scheduling policy or operator scheduling policy into \oursys{} to achieve prioritized scheduling,
since all jobs' batches are submitted to \oursys{} and the operator scheduling can also be achieved via NodeGroup abstract in \oursys{}.
And it's an interesting topic that how to achieve memory sharing efficiency while guaranteeing the job priority.
We leave the consideration of job priority and computation time as a future work.

\textbf{\textit{c) Balancing computation in graph partition and scheduling:}}
% \paragraph{Computation time consideration in graph partition and scheduling}
In graph partition and scheduling of \oursys{},
we only consider balancing the memory footprint of each partition without considering the balance of computation.
In general, when both jobs have high GPU computation demand and can fully utilize the GPU,
imbalanced computation of node groups will have little impact on performance.
However, if one of the models has lower computational demands,
imbalanced computation will degrade the performance to some extent.
However, a significant challenge in taking the computation demand into account is that
parallel kernel scheduling in NVIDIA GPU is opaque.
Therefore, it is difficult to predict whether the kernels in question can be parallelized and assess the performance of parallelized kernels.
As a result, it is challenging to determine if a particular partition and scheduling policy can achieve higher training throughput.
Novel methods need to be developed to overcome these difficulties, which is part of our future research plan.
% A major obstacle to take computation time into account
% is that the parallel kernels scheduling in NVIDIA GPU is a black box concurrently,
% which means it's difficult to predict whether the specific kernels can be parallelized executing
% or the performance of the parallelized kernels.
% Therefore, it's hard to judge if a specific partition and scheduling policy could
% achieve the best overall training throughput.

\textbf{\textit{d) Security:}}
% \paragraph{Security}
Security is crucial for DL training as this process could last long for months,
an accident exiting due to unpredictable faults will waste expensive GPU resources.
When jobs are running in their own CUDA context on the same GPU,
the fault only affects themselves, not to other jobs.
However, in order to share the GPU memory resources across jobs,
they need to run in the same CUDA context,
which will break the \emph{fault isolation}:
% one jobs's fault will lead all other co-located jobs exiting accidentally too.
the failure in a job doesn't impact other co-located jobs.
This security issue exists in the frameworks that merge numerous jobs' contexts into one,
such as NVIDIA MPS~\cite{mps}, Salus~\cite{yu2020salus}, and also \oursys{}.
A good aspect is that DL training job can be recovered at an arbitrary step
as long as the training state (parameters) at that step has been saved.
And this recovery is lightweight since the time to run a single training step is small (seconds level).
But too frequent checkpointing will also influence the overall training performance.
The users need to balance the frequency of checkpointing to make a tradeoff.
% Our work's priority purpose is to improve memory sharing efficiency in concurrent training
% Since share , the concurrent training jobs need to in the same CUDA context
% One job's fault will lead other co-located jobs fail too.
On the other side, another method to support concurrent running of multiple jobs
while guaranteeing fault isolation is virtualizing a GPU to numerous virtualized GPUs (vGPU)~\cite{vGPU}.
The latest technology is Multi-Instance GPU (MIG)~\cite{mig} which is provided by NVIDIA
and supported on the NVIDIA Ampere architecture and after (such as A100 and H100 GPU).
MIG partitions a single GPU into numerous separate GPU instances
where each one owns the dedicated compute, memory, and memory bandwidth resources.
But it has limitations for supporting concurrent training.
First, the compute and memory resource configurations of a GPU instance are pre-defined by NVIDIA that cannot be changed at will.
Besides, when the GPU instances have been configured, the resources cannot be adjusted flexible on-fly.
This collides with the diverse and dynamic resources requirement of DL training jobs.
% The DL training jobs' resource requirements vary much.
Second, the GPU memory resources are isolated across instances, thus cannot be shared.
% It's more suitable for AI inference tasks whose .
% But the optional configurations are limited.
% The compute and memory resources can not be adjusted flexible when the virtualized has been configured.
% \paragraph{Memory fragmentation}
% The next batch of a job won’t start until current batch has finished due to batch dependency.
% Therefore, all memory will be freed and coalesced to a large chunk naturally
% when current parallel running batches complete if there are only two jobs.
% When there are more than two jobs, we flush the pipeline when the parallel running
% batches of these jobs complete to avoid memory fragmentation.
% The memory fragmentation in concurrent training is an interesting topic to research.